\section{STABILITY ANALYSIS}

All four methods have different stability and convergence properties.
We derived a stability criterion for coupled simulations of conjugate heat transfer, based on the relations between temperature and heat flux at the interface of solid and fluid domain.\\
Considering a one dimensional conjugate heat transfer problem, we specify a solid temperature at one boundary of the solid domain and the 
temperature of the fluid. The problem is to find the temperature $ T_{wall} $ and the heat flux $ q_{wall} $ at the interface. With the heat transfer coefficient $ h $, the thermal conductivity $ \lambda_{s} $ and the solid domain width $ L $ it is defined by:
\vskip-.6cm
\begin{eqnarray}
q_{wall} = \frac{\lambda_s}{L}(T_s - T_{wall}) \hspace{20mm} \text{on } \Omega_s, \\
q_{wall} = h(T_{wall} - T_{fluid})  \hspace{20mm} \text{on } \Omega_f, \nonumber
\end{eqnarray}
representing heat fluxes resulting from the solid domain ($ \Omega_s $) and the fluid domain ($ \Omega_f $) computations. Heat fluxes equal at the interface and we have:
\vskip-.6cm
\begin{eqnarray}
\frac{\lambda_s}{L}(T_s-T_{wall}) = h(T_{wall}-T_{fluid}).
\end{eqnarray}
With the Biot number $ Bi=\frac{hL}{\lambda_s} $ the previous equation becomes:
\vskip-.6cm
\begin{eqnarray}
T_s-T_{wall}=Bi(T_{wall}-T_{fluid}).
\end{eqnarray}
Thus, the wall temperature is:\\
\vskip-.6cm
\begin{eqnarray}
T_{wall} = \frac{Ts+Bi \cdot T_{fluid}}{1+Bi}.
\end{eqnarray}
In the next sections, we continue with the stability analysis for each of the four methods.

\subsection{STABILITY OF THE FFTB METHOD}

In the FFTB method, a wall temperature $ T_{wall}^{0} $ is imposed to the fluid domain that varys by a value $ \alpha_0 $ from the correct wall temperature:
\vskip-.6cm
\begin{eqnarray}
T_{wall}^0 = T_{wall} + \alpha_{0}.
\end{eqnarray}
The heat flux solid is then:
\vskip-.6cm
\begin{eqnarray}
q_{wall}^0 = h(T_{wall}^0 - T_{fluid})\\
= h(T_{wall}-T_{fluid})+h\cdot \alpha_{0}\nonumber \\
= q_{wall} +h\alpha_{0} .\nonumber
\end{eqnarray}
Using the new heat flux imposed to the solid results in a new wall temperature:
\vskip-.6cm
\begin{eqnarray}
T_{wall}^{1}=T_s-\frac{L}{\lambda_{s}} \cdot q_{wall}^0\\
= T_s - \frac{L}{\lambda_{s}} \cdot q_{wall} - \alpha^{0} \frac{hL}{\lambda_{s}}\nonumber \\
= T_{wall} - \alpha_{0} \cdot Bi.
\end{eqnarray}
At the i-th iteration, the temperature is given by:
\vskip-.6cm
\begin{eqnarray}
T_{wall}^{i}=T_{wall}+\alpha_{0} \cdot (-Bi)^{i}.
\end{eqnarray}
The heat flux results in:
\vskip-.6cm
\begin{eqnarray}
q_{wall}^{i}=q_{wall}+\alpha_{0} \cdot (-Bi)^{i} \cdot h.
\end{eqnarray}
As we can see, convergence is only achieved, if $ |Bi| < 1 $.

\subsection{STABILITY OF THE TFFB METHOD}

In the TFFB method, a wall temperature $ T_{wall}^{0} $ is imposed to the solid domain that varys by a value $ \alpha_0 $ from the correct wall temperature:
\vskip-.6cm
\begin{eqnarray}
T_{wall}^0 = T_{wall} + \alpha_{0}.
\end{eqnarray}
The heat flux is then:
\vskip-.6cm
\begin{eqnarray}
q_{wall}^0 = \frac{\lambda_{s}}{L}(T_{s} - T_{wall}^0)\\
= \frac{\lambda_{s}}{L}(T_{s} - T_{wall})+\frac{\lambda_{s}}{L} \cdot \alpha_{0}\nonumber \\
= q_{wall} +\frac{\lambda_{s}}{L}\alpha_{0} .\nonumber
\end{eqnarray}
Using the new heat flux imposed to the fluid results in a new wall temperature:
\vskip-.6cm
\begin{eqnarray}
T_{wall}^{1}=T_{fluid}+\frac{q_{wall}^{0}}{h}\\
= T_{fluid}+\frac{q_{wall}}{h} - \alpha^{0} \frac{\lambda_{s}}{hL}\nonumber \\
= T_{wall} - \frac{\alpha_{0}}{Bi}.
\end{eqnarray}
At the i-th iteration, the temperature is given by:
\vskip-.6cm
\begin{eqnarray}
T_{wall}^{i}=T_{wall}+\alpha_{0} \cdot \left(-\frac{1}{Bi}\right)^{i}.
\end{eqnarray}
The heat flux results in:
\vskip-.6cm
\begin{eqnarray}
q_{wall}^{i}=q_{wall}+\alpha_{0} \cdot \left(-\frac{1}{Bi}\right)^{i} \cdot \frac{\lambda_{s}}{L}.
\end{eqnarray}
As we can see, convergence is only achieved, if $ |Bi| > 1 $.

\subsection{STABILITY OF THE hFTB METHOD}

\subsection{STABILITY OF THE hFFB METHOD}